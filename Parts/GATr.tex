\begin{frame}{GATr - Introduzione}
    In diverse situazioni si ha a che fare con dati di natura geometrica, soprattutto
    in ambiti scientifici ed ingegneristici. Il vantaggio di utilizzare dati geometrici
    risiede nel comportamento definito dei dati sotto trasformazioni ben definite 
    (come distanze e rotazioni).

    Con l'obiettivo di utilizzare nel modo migliore questi dati, si introduce il 
    Geometric Algebra Transformer (GATr), un'architettura di rete general-purpose 
    che sfrutta dati geometrici. 
\end{frame}

\begin{frame}{GATr - Idee Chiave}
    GATr riunisce tre idee fondamentali: 
    \begin{itemize}
        \item Algebra Geometrica;
        \item Equivarianza;
        \item Trasformer.
    \end{itemize}
\end{frame}

\begin{frame}{Algebra Geometrica}
    GATr consente di rappresentare i dati come multivettori dell'algebra geometrica 
    proiettiva \(\mathbb{G}_{3,0,1}\), che estende lo spazio vettoriale 
    \(\mathbb{R}^{3}\) a multivettori a 16 dimensioni, capaci di rappresentare vari 
    tipi geometrici e pose \(E(3)\).
\end{frame}

\begin{frame}{Equivarianza}
    GATr è progettato per essere equivariante rispetto al gruppo di simmetria \(E(3)\), 
    che descrive le trasformazioni nello spazio tridimensionale. 

    A tale scopo, sono state sviluppate diverse nuove primitive 
    \(E(3)\)-equivarianti, tra cui mappe lineari equivarianti, un meccanismo di 
    attenzione, non-linearità e strati di normalizzazione.
\end{frame}

\begin{frame}{Trasformer}
    Per il GATr si è scelto di utilizzare l'architettura Transformer grazie alle sue 
    favorevoli proprietà di scalabilità, espressività, addestrabilità e versatilità.
\end{frame}
