\begin{frame}
    \Huge Geometric Algebra
    \rule{\textwidth}{.25pt}
    \small{Autore: Riccardo Lo Iacono}
\end{frame}
\begin{frame}{Algebra di Clifford}
    Sia fissato uno spazio vettoriale \(\mathbb{R}^{n}\), e sia 
    \(\{\Vb{e}_{1}, \ldots, \Vb{e}_{n}\}\) una sua base ortonormale.

    Da questi è possibile definire un nuovo spazio vettoriale,
    detto spazio di Clifford n-dimensionale (\(\Clifford\)).
    Nello specifico, \Clifford rappresenta l'insieme di tutti i 
    possibili sottospazi \(k\)-dimensionali, con \(k \le n\), 
    di tutte le possibili combinazioni dei vettori base.
\end{frame}
\begin{frame}{Multivettori: il caso bi- e tri-dimensionale}
    Si consideri lo spazio \(\mathbb{R}^{2}\) e sia \(\{\Vb{e}_{1}, 
    \Vb{e}_{2}\}\) una sua base ortonormale. 
    \'E noto che comunque presi \(\Vb{a}, \Vb{b} \in \mathbb{R}^{2}\),
    questi possano essere intesi come opportune combinazioni lineari dei vettori 
    base. Ossia
    \[
        \Vb{a} = \alpha_{1}\Vb{e}_{1} + \alpha_{2}\Vb{e}_{2} 
        \qquad \Vb{a} = \beta_{1}\Vb{e}_{1} + \beta_{2}\Vb{e}_{2}
    \]
    inoltre, sappiamo che 
    \begin{equation}\label{eq:1}
        \begin{aligned}
        \Vb{ab} & = 
            (\alpha_{1}\Vb{e}_{1} + \alpha_{2}\Vb{e}_{2})
            (\beta_{1}\Vb{e}_{1} + \beta_{2}\Vb{e}_{2}) \\
            & = \alpha_{1}\beta_{1}\Vb{e}_{1}\Vb{e}_{1} +
                \alpha_{1}\beta_{2}\Vb{e}_{1}\Vb{e}_{2} +
                \alpha_{2}\beta_{1}\Vb{e}_{1}\Vb{e}_{2} + 
                \alpha_{2}\beta_{2}\Vb{e}_{2}\Vb{e}_{2}.
        \end{aligned}\
    \end{equation}
\end{frame}
\begin{frame}
    Defininedo i seguenti assiomi:
    \begin{enumerate}
        \item \(\Vb{e}_{i}\Vb{e}_{i} = 1\)
        \item \(\Vb{e}_{j}\Vb{e}_{i} = - \Vb{e}_{i}\Vb{e}_{j}\)
    \end{enumerate}
    \emph{Equazione \eqref{eq:1}} può essere riscritta come 
    \begin{equation}\label{eq:2}
        \Vb{ab} = (\alpha_{1}\beta_{1} + \alpha_{2}\beta_{2}) + 
            (\alpha_{1}\beta_{2} - \alpha_{2}\beta_{1})\Vb{e}_{1}\Vb{e}_{2}
    \end{equation}
\end{frame}
\begin{frame}
    Da \emph{Equazione \eqref{eq:2}} segue che,
    \(\forall \Vb{a}, \Vb{b} \in \mathbb{R}^{2}\)
    il loro prodotto risulti essere la somma di un termine scalare
    (\(\alpha_{1}\beta_{1} + \alpha_{2}\beta_{2}\)) e da un termine 
    \((\alpha_{1}\beta_{2} - \alpha_{2}\beta_{1})\Vb{e}_{1}\Vb{e}_{2}\).

    Dando un'interpretazione geometrica, 
    \((\alpha_{1}\beta_{2} - \alpha_{2}\beta_{1})\) descrive l'area 
    del rettandolo definita dai vettori \(\Vb{e}_{1}, \Vb{e}_{2}\);
    segue che (\(\Vb{e}_{1}, \Vb{e}_{2}\)) definiscono il piano
    su cui giace l'area.

    In conclusione, \(\Vb{e}_{1}\Vb{e}_{2}\) rappresenta un'area 
    orientata in \(\mathbb{R}^{2}\) ed è definita \emph{bivettore}.
\end{frame}
\begin{frame}
    Un ragionamento analogo può essere fatto per \(\mathbb{R}^{3}\).

    Sia \{\(\Vb{e}_{1}, \Vb{e}_{2}, \Vb{e}_{3}\)\} una base ortonormale
    di \(\mathbb{R}^{3}\). 
    Considerati \(\Vb{a}, \Vb{b} \in \mathbb{R}^{3}\),
    questi saranno della forma 
    \[
        \Vb{a} = \alpha_{1}\Vb{e}_{1} + \alpha_{2}\Vb{e}_{2} 
            + \alpha_{3}\Vb{e}_{3}  
            \qquad 
        \Vb{b} = \beta_{1}\Vb{e}_{1} + \beta_{2}\Vb{e}_{2}
            + \beta_{3}\Vb{e}_{3}
     \]
     Considerandone il prodotto, e applicando gli assiomi definti poco sopra,
     segue 
     \[\begin{aligned}
         \Vb{ab} & = (\alpha_{1}\beta_{1} + \alpha_{2}\beta_{2} + \alpha_{3}\beta_{3}) \\ 
            & + (\alpha_{1}\beta_{2} - \alpha_{2}\beta_{1})\Vb{e}_{1}\Vb{e}_{2} \\
            & + (\alpha_{1}\beta_{3} - \alpha_{3}\beta_{1})\Vb{e}_{1}\Vb{e}_{3} \\
            & + (\alpha_{2}\beta_{3} - \alpha_{3}\beta_{2})\Vb{e}_{2}\Vb{e}_{3} \\
     \end{aligned}\]
\end{frame}
\begin{frame}
    Oltre le tre aree, in \(\mathbb{R}^{3}\) è possibile definire un elemento 
    dato dal prodotto dei vettori base: \(\Vb{e}_{1}\Vb{e}_{2}\Vb{e}_{3}\).

    Similarmente all'area orientata rappresentata da \(\Vb{e}_{1}\Vb{e}_{2}\),
    \(\Vb{e}_{1}\Vb{e}_{2}\Vb{e}_{3}\) rappresenta un volume orientata in 
    \(\mathbb{R}^{3}\).
\end{frame}
\begin{frame}{Multivettori: il caso generale}
    In generale fissato un n, e assunti \{\(\Vb{e}_{1}, \ldots, \Vb{e}_{n}\)\}
    una base ortonormale di uno spazio \(\mathbb{R}^{n}\),
    un multivettore \(\Vb{a} \in \mathbb{R}^{n}\) è un elemento della forma
    \[
        \Vb{a} = \alpha_{0} + \alpha_{1}\Vb{e}_{1} + \cdots \alpha_{n}\Vb{e}_{n}
            + \alpha_{1,2}\Vb{e}_{1}\Vb{e}_{2} + \cdots 
            + \alpha_{1, \cdots, n}\Vb{e}_{1} \cdots \Vb{e}_{n}
    \]
    e lo spazio che contiene tutti questi multivettori è detto spazio di Clifford
    n-dimensionale (\Clifford). 
    In ultimo, il multivettore dato dal prodotto di ogni vettore base è detto
    \emph{pseudo-scalare}.
\end{frame}
\begin{frame}{Operazioni tra multivettori}
    Fissata una qualche algebra di Clifford \Clifford, 
    su gli elementi della stessa è possibile applicare diverse operazioni,
    quali 
    \begin{itemize}
        \item prodotto geometrico
        \item prodotto interno
        \item operazioni unarie
    \end{itemize}
\end{frame}
\begin{frame}{Operazioni tra multivettori: il prodotto geometrico}
    Sia considerata \Clifford[2], (l'estensione al caso n-simo è immediata),
    e siano \(\Vb{a}, \Vb{b}\) due multivettori.

    Il prodotto geometrico tra i due consiste nel moltiplicare ciascuna delle 
    componenti del primo multivettore per quelle del secondo, 
    tenendo conto dei seguenti assiomi:
    \begin{itemize}
        \item \(\Vb{e}_{i}\Vb{e}_{i} = \pm 1\)
        \item \(\Vb{e}_{i}\Vb{e}_{j} = -\Vb{e}_{i}\Vb{e}_{j}\)
        \item \(\lambda\Vb{e}_{i} = \Vb{e}_{i}\lambda\)
    \end{itemize}
\end{frame}
\begin{frame}
    Si ha cioè che 
    \[\begin{aligned}
        \Vb{ab} &= 
            (\alpha_{0} + \alpha_{1}\mathbf{e}_{1} + \alpha_{2}\mathbf{e}_{2} 
                + \alpha_{12}\mathbf{e}_{12})
            (\beta_{0} + \beta_{1}\mathbf{e}_{1} + \beta_{2}\mathbf{e}_{2}
                + \beta_{12}\mathbf{e}_{12}) \\
            & = \alpha_{0}\beta_{0} + \alpha_{0}\beta_{1}\mathbf{e}_{1}
            + \alpha_{0}\beta_{2}\mathbf{e}_{2} 
            + \alpha_{0}\beta_{12}\mathbf{e}_{12} \\
            & + \alpha_{1}\beta_{0}\mathbf{e}_{1} + \alpha_{1}\beta_{1}\mathbf{e}_{11}
              + \alpha_{1}\beta_{2}\mathbf{e}_{12} 
              + \alpha_{1}\beta_{12}\mathbf{e}_{112} \\
            & + \alpha_{2}\beta_{0}\mathbf{e}_{2} + \alpha_{2}\beta_{1}\mathbf{e}_{21}
              + \alpha_{2}\beta_{2}\mathbf{e}_{22} 
              + \alpha_{2}\beta_{12}\mathbf{e}_{212} \\ 
            & + \alpha_{12}\beta_{0}\mathbf{e}_{12} 
              + \alpha_{12}\beta_{1}\mathbf{e}_{121}
              + \alpha_{12}\beta_{2}\mathbf{e}_{122} 
              + \alpha_{12}\beta_{12}\mathbf{e}_{1212}  
    \end{aligned}\]
    diventa, applicando gli assiomi di cui sopra
    \[\begin{aligned}
        \Vb{ab} & = 
            (\alpha_{0}\beta_{0} + \alpha_{1}\beta_{1} 
                + \alpha_{2}\beta_{2} - \alpha_{12}\beta_{12}) \\ 
            & + (\alpha_{0}\beta_{1} + \alpha_{1}\beta_{0} 
                - \alpha_{2}\beta_{12} + \alpha_{12}\beta_{2})\Vb{e}_{1} \\
            & + (\alpha_{0}\beta_{2} + \alpha_{2}\beta_{0} 
                + \alpha_{1}\beta_{12} - \alpha_{12}\beta_{1})\Vb{e}_{2} \\
            & + (\alpha_{0}\beta_{12} + \alpha_{12}\beta_{0} 
                + \alpha_{1}\beta_{2} - \alpha_{2}\beta_{1})\Vb{e}_{12}
    \end{aligned}\]
\end{frame}
\begin{frame}
    Si dimostra inoltre che il prodotto geometrico gode delle seguenti proprietà:
    \begin{itemize}
        \item [ ] \emph{Associatività}: 
            \[
                \forall \Vb{a}, \Vb{b}, \Vb{c} \in \mathbb{R}^{n},
                (\Vb{ab})\Vb{c} = \Vb{a}(\Vb{bc})
            \]
        \item [ ] \emph{Commutatività rispetto uno scalare}: 
            \[
                \forall \Vb{a} \in \mathbb{R}^{n}, \lambda \in \mathbb{R}, 
                \lambda\Vb{a} = \Vb{a}\lambda
            \]
        \item [ ] \emph{Distributività rispetto la somma}:
            \[
                \forall \Vb{a}, \Vb{b}, \Vb{c} \in \mathbb{R}^{n},
                \Vb{a}(\Vb{b} + \Vb{c}) = \Vb{ab} + \Vb{ac}
            \]
    \end{itemize}
\end{frame}
\begin{frame}{Operazioni tra multivettori: prodotto interno}
    Sebbene ne esistano varie formulazioni, 
    di interesse è la forma contratta di Lounesto (\(\rfloor\) ). 
    Nello specifico, questa può essere intesa come una generalizzazione del 
    prodotto scalare. 

    Siano \(\alpha, \beta\) scalari, \(\Vb{a}, \Vb{b}\) vettori e \(\Vb{A}, 
    \Vb{B}, \Vb{C}\) multivettori, allora vale quanto segue:
    \[\begin{aligned}
        \alpha \rfloor \beta & = \alpha\beta \\
        \alpha \rfloor \Vb{b} & = \alpha\Vb{b} \\ 
        \Vb{a} \rfloor \beta & = 0 \\ 
        \Vb{a} \rfloor \Vb{b} & = \Vb{ab} \\ 
        \Vb{a} \rfloor (\Vb{b} \wedge \Vb{C}) & = 
            (\Vb{a} \rfloor \Vb{b}) \wedge \Vb{C} - \Vb{b} \wedge (\Vb{a} \rfloor \Vb{C})
    \end{aligned}\]
\end{frame}
\begin{frame}{Operazioni tra multivettori: il prodotto geometrico v2}
    Formulazione alternativa del prodotto geometrico è la seguente:
    \[
        \Vb{ab} = \Vb{a} \cdot \Vb{b} + \Vb{a} \wedge \Vb{b}
    \]
    ove \(\Vb{a} \cdot \Vb{b}\), detto \emph{prodotto inteno},
    determina la componente scalare del prodotto; 
    \(\Vb{a} \wedge \Vb{b}\), detto \emph{prodotto esterno},
    ne determina la componente multivettoriale.

    Più correttamente sono \(\Vb{a} \cdot \Vb{b}\) e \(\Vb{a} \wedge \Vb{b}\)
    ad essere definiti in termini di prodotto geometrico. 
    Nello specifico
    \[\begin{aligned}
        \Vb{a} \wedge \Vb{b} = \frac{1}{2}(\Vb{ab} - \Vb{ba}) \\ 
        \Vb{a} \cdot \Vb{b} = \frac{1}{2}(\Vb{ab} + \Vb{ba})
    \end{aligned}\]
\end{frame}
\begin{frame}{Operatori unari}
    Di interesse risultano essere l'operatore duale (\(\Vb{x}^{*}\))
    e l'inverso (\(\Vb{x}^{-1}\)).

    Sia considerato l'inverso: questi, dato \(\Vb{a}\) un multivettore,
    definisce il multivettore \(\Vb{a}^{-1}\) tale che \(\Vb{a}\Vb{a}^{-1} = 1\).
    Come nel caso matriciale, l'esistenza di \(\Vb{a}^{-1}\) non è garantita.

    Sono per tanto considerati i cosidetti versori: 
    multivettori esprimibili nella forma \(\Vb{a} = v_{1}v_{2}\cdots v_{k}\);
    per i quali è possibile garantire l'esistenza dell'inverso.
\end{frame}
\begin{frame}
    Considerando il duale: sia \(\Vb{a}\) un multivettore, 
    il suo \emph{duale} \(\Vb{a}^{*} = \Vb{a}\Vb{I}^{-1}\) rappresenta il vettore normale al multivettore.

    Sia ad esempio considerato \(\Vb{e}_{12}\), calcolandone il duale si ha
    \[\begin{aligned}
        \Vb{e}_{12}^{*} & = \Vb{e}_{1}\Vb{e}_{1}\Vb{e}_{3}\Vb{e}_{2}\Vb{e}_{1} \\ 
            & = - \Vb{e}_{1}\Vb{e}_{3}\Vb{e}_{2}\Vb{e}_{2}\Vb{e}_{1} \\ 
            & = - \Vb{e}_{1}\Vb{e}_{3}\Vb{e}_{1} \\ 
            & = \Vb{e}_{3}\Vb{e}_{1}\Vb{e}_{1} \\ 
            & = \Vb{e}_{3}
    \end{aligned}\]
\end{frame}
\begin{frame}{Bibliografia}
    \begin{itemize}
        \item Jaap Suter, \emph{Geometric Algebra Primer}, 2003
        \item Alan Macdonald, \emph{A Survey of Geometric Algebra and Geometric Calculus,
            applied Geometric Algebra in Computer Science and Engineering}, Barcelona, 
        Spain, 29-31 Luglio, 2015

        \item Sivia Franchini, Giorgio Vassallo, Filippo Sorbello, 
            \emph {A brief introduction to Clifford algebra},
            Febbraio 2010.
    \end{itemize}
\end{frame}
